\documentclass[10pt]{article}
\usepackage[margin=0.5in]{geometry}

\title{A Primer on Analysis, Topology, and Measure Theory}
\author{Aditya Sengupta}

\usepackage{analysis}
\usepackage{subfiles}
\let\oldsection\section
\renewcommand\section{\clearpage\oldsection}

\begin{document}
    \maketitle

    This is a document with multiple purposes:

\begin{enumerate}
    \item To relate measure theory content from Math 202A to probability content from EE 126.
    \item To act as an easy reference for the Math 104 content that's used in Math 202A.
    \item As a replacement for lecture notes for Math 202A, because I don't understand what's going on there right now.
\end{enumerate}

I'm going to try and keep this compact. %In other words, this document is going to be a finite subcover of Math 104 and Math 202A (hopefully, by the end, you'll find that hilarious.) 
This means that sometimes, I'll have a definition that's somewhat opaque, or dependent on a lot of terms that were just introduced (otherwise this would be 200 pages and it'd be scary to even start reading.) I'll try and illustrate those with examples, or provide a restatement in simpler words, whenever that happens.    

Section 1 consists of mathematical preliminaries: basic notation, and the definition of an equivalence relation (because it comes up in some surprising places). Sections 2 and 3 runs the reader through the essential parts of real analysis to understand metric spaces, in what I hope are relatively easy terms. Section 4 is an introduction to metric spaces, Section 5 covers topology, and Section 6 covers measure theory.

    \tableofcontents
    \let\tableofcontents\relax

    \subfile{sections/preface.tex}
    \subfile{sections/construction_of_R.tex}
    \subfile{sections/real_analysis.tex}
    \subfile{sections/metric_spaces.tex}
    \subfile{sections/topology.tex}
    \subfile{sections/measure_theory.tex}
    \subfile{sections/appendix.tex}

\end{document}